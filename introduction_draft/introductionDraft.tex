\documentclass{article} 
\usepackage[utf8]{inputenc} 
\usepackage{cite}
% \usepackage[margin=1.5in]{geometry}
\title{Simulation of Stellar Streams in the Milky Way} 
\author{Peixian Wang} 
\date{February 2016}

\begin{document}

\maketitle

\section{Introduction}

\indent The Milky Way's vast arrangement of stars is broadly categorized into many different categories: 4 galactic quadrants, the galactic center, the spiral arms, and the halo. However, research have uncovered more aspects of the Milky Way, particularly regarding the rings composed of stars, gas, and dust around the Milky Way, known as stellar streams. 

\subsection{Discovery} \indent First predicted by Lynds and Toomre \cite{lyndsAndToomre} in their model for estimating the Cartwheel galaxy (ESO 350-40), Lynds and Toomre found that the mass dispersion of the Cartwheel galaxy's rings results from kinematic epicyclic oscillations after an impulsive disturbance occurs. From this Curtis and Lotan \cite{symmetricStellarRings} found that by applying singularity theory techniques to the kinematic radial oscillation model that the galactic ring densities matched with their simulations of galactic cannibalization. 

\indent Since the orbits of all celestial objects are affected by some form of orbital resonance, stellar streams find themselves no exception. Within the region in which the resonances occur, the collision of gas clouds excite star formation, causing this region to be extremely bright. By using the inverse of this relationship, it becomes possible to find galactic resonances using rings and pseudorings \cite{ringsAndPseudoRings}. Since fine grain IIIa-J emulsion makes it possible to spot the faint and small stellar streams, it was found that the stellar streams orbiting over 1000 early to intermediate Hubble type spiral galaxies within the southern hemisphere do follow the Linblad resonance. From this, it becomes possible to infer the rough orbits of possible galaxies that lie within the resonances. 

\indent A subset of these stellar streams are polar rings, which similarly consist of dust, stars, and gas. NGC 660 is a popular example, with it's polar rings roughly aligned in its minor axis. These minor polar rings are found to consist of mostly neutral hydrogen, along with a ratio of velocity distribution from the central bulge to the circular velocity being abnormally high \cite{distributionOfAtomicHydrogen}. These polar rings also contained blue stars and several billions of solar masses in gas, despite the gas-deprived host galaxy, suggesting that the polar rings were also formed by accretion or cannibalization from other companions relatively recently. The shape of the polar rings are curious however, since the tilted structure, relative flatness, and precession speed all suggest that the shape of the polar ring should be not exist than a Hubble time \cite{selfGravitatingPolarRings}. Yet the existence of some polar rings that maintain their shape but also contain redder stars refute this notion, thereby suggesting that somehow the rings themselves have settled, or forced into settling, into a stable equilibria with the gravitational field. Struck-Marcell and Lotan found that a quasi-impulsive kinematic oscillation equations provide accurate models into the ring structure and a $N$-body simulation is sufficient to model the shape and dynamics of these rings. 

\subsection{Composition \& Formation} \indent $N$-body simulations have previously strongly suggested that the galactic merging of smaller galaxies is a main cause for galaxy formation \cite{galaxyNBody}, which strongly suggest that stellar streams are a result galaxy's galactic tide cannibalizing on nearby satellite galaxies. This mechanism only affects the satellite galaxy due to the relatively weak gravitational fields, but by using fossil signature of ancient accretion events within the Milky Way halo, Johnston et al. \cite{accretionEvents} found that if the satellite galaxy is very small, the debris trails remain aligned to the parent satellite galaxy's orbit, for the lifetime of the host galaxy. These orbits become extremely close to circular once the orbit is near planar. If the mass of the satellite galaxy is greater than one ten-thousandth of the host galaxy, then the stellar stream produced in the process has two non-equidistant trailing points where the attractive force of the host halo and satellite are at zero-velocity. As a result of this, stellar stream that is created begins to be effected by the satellite's own self-gravity \cite{dymanicsOfTidalTails}, and begins to be accelerated into many different directions. These directions vary based on the mass of the satellite and host, along with the structure of the halo. Due to this, the radial velocity of stars left in the stellar stream will vary proportionally to the satellite galaxy's mass, which then means that all models for stellar streams would need to be mass-dependent modeling. 

\subsection{Stellar Streams within the Milky Way} 
\subsubsection{General Properties} \indent As the project focuses mainly on stellar streams created with Milky Way and Milky Way satellite galaxy conditions, we will focus the bulk of the information on these conditions. In 2002, a peculiar group of F/G stars with $B-V \leq 0.7$ and $V$-band apparent magnitudes in the range 17.0-19.5 were found between 30$^\circ$ above and 45$^\circ$ below the Galactic plane, spanning a distance of 5 kpc \cite{lastMajorInvasion}. These stars had a mean rotation speed of ~100$km/s^{-1}$, 80$km/s^{-1}$ off from the predicted ~180$km/s^{-1}$. Retaining kinematic signatures that distinguish themselves from the other stars within the think disk. This group of stars were identifiable due to the asymmetry in the number of stars in the thin disk and inner halo on one side of the galactic center. As a result this group of stars was theorized to be a result of the a satellite galaxy that merged with the Milky Way, possibly forming the thick disk in the process. This process also led to conclusions that stars within stellar streams would be distributed with lower net rotational streaming than the rest of the disk. The same kinematic signatures can be applied in a similar manner to the halo, many moving groups and halo substructures have been found with relatively similar kinematic signatures, suggesting that the halo was similarly affected by the satellite galaxies. 

\indent The idea of satellite galaxies interacting with the Milky Way in a non-trivial way is not foreign either: the orbital path of the Sagittarius dwarf spheroidal galaxy is highly correlated with at least four of the globular clusters within close proximity to the Milky Way. These four globular clusters are distributed along the path of the Sagittarius Stream, along with the observational evidence that stars within the Sagittarius galaxy are being lost to the Milky Way due to the Milky Way's tidal force, led to conclusions that these four clusters were originally former members of the Sagittarius galaxy \cite{globlarClusters}.

\subsubsection{Virgo Overdensity \& the Sagittarius Stream} \indent The Virgo overdensity and Sagittarius Stream are given special attention due to their secularity and uniqueness. Found in 2000 \cite{ghostOfSagittarius} using accurate multi-color Sloan Digital Sky Survey (SDSS) by probing structures in the Milky Way halo with distances in the range of 2-150 kpc, two branches of the Sagittarius stream were found, with stars having significantly bluer turnoff. Since the Sagittarius galaxy has an orbital path that is high inclined with respect to the disk inclination, it crosses the Milky Way near the Virgo constellation. 

\indent The Virgo constellation hosts a clump of 21 RR Lyrae stars (henceforth referred to as the Virgo overdensity) which has a considerably smaller line of sight depth than the RR Lyrae overdensity at the epicenter of the Sagittarius stream \cite{virgoOverdensity}. Following the construction of a three-dimensional density distribution, the Virgo overdensity shows a giant clump of stars crossing the Galactic plane orthogonally and even extending into the souther Galactic hemisphere. Martinez-Delgado et al, through a $N$-body simulation of one of the arms belonging to the Sagittarius galaxy, found that the properties of the Virgo overdensity matches the theoretical models of the Sagittarius stream, suggesting that the Virgo overdensity is actually an arm of the Sagittarius galaxy falling into the Milky Way disk. 

\subsubsection{Other Streams within the Milky Way}
\indent Beyond the Sagittarius stream, many other streams exist within the Milky Way, with notable ones such as the Arcturus Stream, Magellanic Stream, and Monoceros ring. Near the star Arcturus, the Arcturus Stream was found as a result of similar proper motion, containing older stars lacking in heavy elements \cite{ghostOfSagittarius}. The Magellanic Stream consists of large gas clouds, first found in 1965. Stretching to over 180 kpc, the gas moves extremely fast \cite{magellanicStream}. The Monoceros Ring is a result of the Canis Major overdensity, a ring spanning over 200,000 light years, wrapping around the Milky Way over three times \cite{monocerosRing}. 

\subsection{Research Potential \& Impact}
\indent Research in stellar streams lends a great amount of insight into the early formation of galaxies. Knowing about the formation of the Sagittarius stream has already lead to insights on the structure of the Milky Way halo, further simulations of stellar streams can indicate the status of various satellite galaxies under tidal force.

\bibliography{intro} 
\bibliographystyle{unsrt} \end{document}
