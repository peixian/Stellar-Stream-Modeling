\documentclass{article}
\usepackage[utf8]{inputenc}
\usepackage[margin=0.5in]{geometry}
\title{Formation of Rings through Galactic Collisions}
\author{pwang16 }
\date{February 2016}

\begin{document}

\maketitle

\section{Introduction}
Stellar streams around galaxies are often occurrences, with many orbiting the Milky Way, such as the the Sagittarius Stream, thought to be formed from tidally stripped stars from the Sagittarius Galaxy. Newberg et al. reported a low latitude stream of stars around the halo in the Milky Way. With the velocity dispersion of the stars matching in similarity to the dispersion of stars in the Sagittarius dwarf galaxy tidal stream, they concluded that this stream was created through a collision of the Milky Way and Sagittarius dwarf galaxy. 
This Sagittarius stream was also mapped by Delgando et al, who used the Virgo overdensity as the basis in their $N$-body simulation to model the Sagittarius stream.
The Andromeda galaxy also has a very well known stellar stream composed of metal rich stars (Ibata et al.) called the M31 giant stellar stream, thought to be created through the cannibalization of nearby satellite galaxies. This stream contains a variety of loops and ripples, suggesting further influence after it was formed from Andromeda's other rings. From the Milky Way and Andromeda galaxies  the rough creation process of these stellar streams are well known, the influence of the impact on the shape of the ring is less well known. 

\section{Simulation Plan}
Using Enzo as a starting code, a galaxy with similar morphology to the Milky Way will be created. This galaxy will be a disk galaxy with about $2kly$ thickness and mass of $~1.5*10^{12}  M_{\odot}$ A variety of smaller satellite galaxies will be created to orbit around the Milky Way, with similar characteristics to the Sagittarius Dwarf, Sculptor Dwarf, and Sextans Dwarf. By varying the impact speed, angle, and distance between the dwarf galaxies and the Milky Way, the simulation will explore the interactions between the two galaxies. Given enough time, the experiment will be expanded out to using a variety of galaxies with morphologies different than the Milky Way and satellite galaxies different to the ones existing currently. 

\section{Planned Analysis}
The analysis of the simulation will consist of five stages for each individual host galaxy ,satellite galaxy, and impact conditions configuration: initial conditions, pre-full impact, full impact, short-term results, and long-term results. The initial conditions will be an analysis of the pre-existing galaxies, while the pre-full impact will be an analysis on when the satellite galaxy reaches the edge of the host galaxy. The full impact will be an analysis when the satellite galaxy achieves maximum gravitational influence from the host galaxy, while the short-term results will be an analysis on what shortly after the impact, before the simulation converges. The long-term results will be once the simulation has converged to a stable state. Throughout each stage the formation of the stellar stream will be measured, as well as the mass and velocity dispersion of the satellite galaxy. 
\end{document}
